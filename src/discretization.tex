\chapter{Discretization}

The continuous variables have to be discretized in a finite number of elements
in order to be able to solve the problem in a computer. The following increments
are defined:
\begin{description}
	\item[$\Delta t^{n+\frac12}=t_{n+1}-t_n$]: temporal increments,
	\item[$\delta x_{i+\frac12}=x_{i+1}-x_i$]: spatial increments.
\end{description}

We go to describe two classical numerical approach families: finite differences
and finite volumes.

\section{Finite differences}

Finite differences discretizations use the values defined at determined points
and times. The following notation is used:
\begin{description}
	\item[$\vec{u}_i^n=\vec{u}\left(x_i,\,t_n\right)$]: vector of variables
		discretized at a determined point and time.
\end{description}

\section{Finite volumes}

In finite volumes the spatial averages of the variables are defined:
\begin{description}
	\item[$\displaystyle\vec{U}_i^n=\frac1{\delta x}\,
		\int_{x_i-\frac12\,\delta x}^{x_i+\frac12\,\delta x}
		\vec{u}\left(x,\,t_n\right)\,dx$]:
		vector of averaged variables integrated at a time around a point.
\end{description}

\section{Issues and adventages}



